\begin{tabularx}{\textwidth}{lX}	
%
{\bf Keywords:} & Black holes, Scalar fields, Proca fields, Maxwell fields, Hairy black holes, Boson stars, superradiance.\\
\\
%
{\bf Abstract:} & 
This thesis presents recent studies on generalizations of Kerr black holes with scalar hair.
After a short introduction, we will first consider so-called $Q$-clouds which are rotating realizations of solitonic $Q$-balls.
After this, we will move onto fully backreacting solutions where we will consider three different examples: 1) Self-interacting scalar hair around Kerr black holes, 2) Kerr black holes with Proca hair and 3) Kerr-Newman black holes with gauged/ungauged scalar hair.
In each case we will discuss both the similarities and difference as compared to their Kerr black holes with scalar hair.
In addition to this, we also consider the horizonless solitons of the corresponding theory as they play an essential role in describing each hairy black hole.
%
\end{tabularx}

\begin{tabularx}{\textwidth}{lX}	
%
{\bf Palavras-chave:} & Buracos negros, campos escalares, Campos de Proca, campos escalares, campos de Maxwell, Buracos negros peludas (??), estrelas boson (??), super-radi\^ancia.\\
\\
%
{\bf Resumo:} 
% & Nesta tese apresentamos estudos recentes sobre campos escalares e vetoriais de teste, em torno de buracos negros na teoria cl\'assica da relatividade geral. A tese encontra-se dividida em duas partes, de acordo com as propriedades asimt\'oticas do espa\c{c}o-tempo em estudo. \\
% & Na primeira parte, investigamos os campos escalar e de Proca num espa\c{c}o asimt\'oticamente plano. Para o campo de Proca, obtemos um conjunto completo de equa\c{c}\~oes do movimento em espa\c{c}os esfericamente sim\'etricos em dimens\~oes elevadas. Estas equa\c{c}\~oes s\~ao resolvidas numericamente, tanto para o c\'alculo de radia\c{c}\~ao de Hawking como para o c\'alculo de estados quasi-ligados. No primeiro c\'alculo, pela primeira vez, efetuamos um estudo preciso dos graus de liberdade longitudinais que s\~ao induzidos pelo termo de massa do campo. Este estudo pode ser usado para melhorar o modelo da evapora\c{c}\~ao de buracos negros acoplados a campos vetoriais massivos e geradores de eventos de buraco negro usados presentemente no Grande Colisor de H\'adrons para testar modelos de gravidade com dimens\~oes extra \`a escala do TeV. Relativamente aos estados quasi-ligados, encontramos estados com tempos de vida arbitrariamente longos para o campo de Proca carregado, no buraco negro de Reissner-Nordstr\"om. Como compara\c{c}\~ao, obtemos estados com tempos de vida arbitrariamente longos tamb\'em para o campo escalar.\\
% & Na segunda parte da tese, apresentamos investiga\c{c}\~ao sobre instabilidades super-radiantes para os campos escalar e de Maxwell em espa\c{c}o asimt\'oticamente anti-de Sitter. No caso escalar introduzimos um acoplamento de carga entre o campo e o background e mostramos que instabilidades super-radiantes existem para todos os valores do momento angular total, $\ell$, em dimens\~oes mais elevadas. Este resultado
%
\end{tabularx}

