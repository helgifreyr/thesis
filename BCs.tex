\chapter{Boundary conditions and physical quantities}
\label{BCs}
\section{Boundary conditions}
\label{sec_bc}
%%%%%%%%%%%%%%%%%%%%%%%%%%%%%%%%
To obtain KBHsSH solutions, appropriate boundary conditions must be imposed, that we now summarize.
  
At spatial infinity, $r\rightarrow\infty$, we require that the solutions approach a Minkowski spacetime
with vanishing matter fields:
\begin{equation}
  \lim_{r\rightarrow \infty}{F_i}=\lim_{r\rightarrow \infty}{W}=\lim_{r\rightarrow \infty}{\phi}=0\ .
\end{equation}
%
%\item[$\bullet$] 
On the symmetry axis, $i.e.$ at $\theta=0,\pi$, axial symmetry and regularity require that
\begin{equation}
\partial_\theta F_i = \partial_\theta W 0\ .
\end{equation} 
%
Moreover, all solutions herein are invariant under a reflection in the equatorial plane ($\theta=\pi/2$).
The event horizon is located at a surface with constant radial variable, $r=r_H>0$.
By introducing a new radial coordinate $x=\sqrt{r^2-r_H^2}$ 
the boundary conditions and numerical treatment of the problem are simplified.
Then one can write an approximate form of the solution  near the horizon as a power series in $x$,
which implies the following boundary conditions
\begin{equation}
\partial_x F_i \big|_{x=0}= \partial_x \phi  \big|_{x=0} =  0,~~W \big|_{x=0}=\Omega_H,
%\frac{w}{m},~~
\label{bch1}
\end{equation}
where $\Omega_H $ is the horizon angular velocity.
The existence of a smooth solution imposes the
\textit{synchronization condition}
%
\begin{eqnarray}
\label{KNcond}
w=m\Omega_H\ .
\end{eqnarray}
%
 


%%%%%%%%%%%%%%%%%%%%%%%%%%%%%%%%
\section{Physical quantities}
\label{sec_pq}
%%%%%%%%%%%%%%%%%%%%%%%%%%%%%%%%
Axi-symmetry and stationarity of the spacetime Eq. \eqref{eqn:HBH-ansatz}
 guarantee the existence of two conserved global charges, the total mass $M$ and angular momentum $J$, 
which can be computed either as Komar integrals at spatial infinity or, equivalently, 
from the decay of the appropriate metric functions:
%
\begin{eqnarray}
\label{KNasym}
g_{tt} =-e^{2F_0}N+e^{2F_2}W^2r^2 \sin^2 \theta 
\to
 -1+\frac{2GM}{r}+\dots, ~~
g_{\varphi t}=-e^{2F_2}W r^2 \sin^2 \theta
\to 
\frac{2GJ}{r}\sin^2\theta+\nonumber \dots.  
\end{eqnarray}
%
These quantities can be split into the horizon contribution -- computed as a Komar integral on the horizon -- and the matter contributions, composed of the scalar field, computed as the volume integrals of the appropriate energy-momentum tensor components: 
%
\begin{eqnarray}
\label{MH-hor}
M=M^\Psi+M_H\ , \qquad J=J^\Psi+J_H\ ,
\end{eqnarray}
where $M_H$ and $J_H$  are the horizon mass and angular momentum.
$M^\Psi$ and $J^\Psi$ are the scalar field energy and angular momentum outside the horizon,
with  
\begin{align}
\label{Mpsi}
-M^\Psi\equiv  \int_{\Sigma} dS^a (2T_{ab}^\Psi \xi^b-T^\Psi\xi_a)
 = 4\pi \int_{r_H}^\infty dr \int_0^\pi d\theta~r^2\sin \theta ~e^{F_0+2F_1+F_2}
 \left(
 \mu^2-2 e^{-2F_0}\frac{w(w-mW)}{N}
 \right)\phi^2 ~~,
\end{align}
while $J^\Psi=mQ$,
where $Q$ is the Noether charge 
associated with the the $global$ $U(1)$ symmetry of the complex scalar field 
\begin{eqnarray}
\label{Q-int}
Q=4\pi \int_{r_H}^\infty dr \int_0^\pi d\theta 
~r^2\sin \theta ~e^{-F_0+2F_1+F_2}  \frac{(w-mW)}{N}\phi^2 ~.
\end{eqnarray}
To measure the hairiness of a BH, it is convenient 
to introduce the normalized Noether charge
\begin{eqnarray}
\label{KNq}
q=\frac{mQ}{J}~,
\end{eqnarray}
with $q=1$ for solitons and $q=0$ for KN BHs.
 %
The BH horizon introduces a temperature $T_H$ and an entropy $S={A_H}/({4G})$,
where 
%
\begin{eqnarray}
\label{KNTHAH}
T_H=\frac{1}{4\pi r_H}e^{(F_0-F_1)|_{r_H}}\ ,
\qquad
A_H=2\pi r_H^2 \int_0^\pi d\theta \sin \theta~e^{(F_1+F_2)|_{r_H}} \ .
\end{eqnarray}
%
The various quantities above are related by a Smarr mass formula 
%
\begin{eqnarray}
\label{smarr}
M=2 T_H S +2\Omega_H (J-m Q) + M^\Psi,
\end{eqnarray}
The solutions satisfy also the 1st law 
%
\begin{eqnarray}
\label{first-law}
dM=T_H dS +\Omega_H dJ  .
\end{eqnarray}
%
Finally, observe that from \eqref{smarr} and \eqref{MH-hor} are consistent with a different Smarr relation, only in terms of horizon quantities  
%
%
\begin{eqnarray} 
\label{rel-hor}
M_H=2T_H S+2 \Omega_H J_H~.
% + \Phi_H Q_E \ .
\end{eqnarray}
