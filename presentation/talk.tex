\documentclass[xcolor=dvipsnames]{beamer}

\usepackage{t1enc}
\usepackage{etex}
\usepackage{amsmath} 
\usepackage{amsfonts} 
\usepackage{amssymb}
\usepackage{amsthm}
\usepackage{tikz}
\usepackage{mathrsfs}
\usepackage{multicol}
\usepackage{animate,media9}


\mode<presentation>
{
  \usetheme{Pittsburgh}      % or try Darmstadt, Madrid, Warsaw, ...
  \usecolortheme{orchid} % or try albatross, beaver, crane, ...
  \usefonttheme{serif}  % or try serif, structurebold, ...
  \setbeamertemplate{navigation symbols}{}
  % \setbeamertemplate{caption}[numbered]
  \setbeamertemplate{headline}{%
  \leavevmode%
    \hbox{%
      \begin{beamercolorbox}[wd=\paperwidth,ht=5.0ex,dp=1.125ex]{palette quaternary}%
      \insertsectionnavigationhorizontal{\paperwidth}{\hskip0pt plus1filll}{}
      \insertsubsectionnavigationhorizontal{\paperwidth}{\hskip10pt plus1filll}{}
      \end{beamercolorbox}%
   }
  }
} 

\usepackage[english]{babel}
\usepackage[utf8]{inputenc}

\usepackage{pgfpages}
% \pgfpagesuselayout{resize to}[%
%   physical paper width=8in, physical paper height=6in]

\title[]{Kerr black holes with hair}
\author[Helgi,Carlos,Eugen] % (optional, for multiple authors)
{\textbf{Helgi Freyr Rúnarsson}\\[3mm] \small{\and C.~Herdeiro \and E. Radu \and P. Cunha \and J. Delgado}}% \and J. C. Degollado\inst{2}}}
\institute[Universities Here and There] % (optional)
{
  % \inst{1}%
  Department of Physics and CIDMA\\
  Universidade de Aveiro
  % \and
  % \inst{2}%
  % Guadalajara University
}
\date % (optional)
{ENAA 2016, 9th of September, 2016}

\begin{document}
\begin{frame}[plain]
  \titlepage
\end{frame}

\begin{frame}{Black holes}
  \begin{itemize}[<+->]
    \item Black holes most likely exist.
    \item Very heavy, and compact, objects seem to reside at the center of most galaxies.
    \item In the next decade or so, VLBIs will be able to resolve the center of galaxies.
    \item It is therefore an excellent time to consider alternatives to the paradigmatical Kerr black holes of Einstein's general relativity.
  \end{itemize}
\end{frame}

\begin{frame}{Public service announcement}
  \begin{itemize}[<+->]
    \item The solutions I will discuss:
  \begin{itemize}[<+->]
    \item are completely within Einstein's general relativity,
    \item obey all energy conditions,
    \item can provide distinct, and unique, phenomenology.
  \end{itemize}
  \end{itemize}
\end{frame}

\subsection{}
\begin{frame}[plain]{}
  \tableofcontents
\end{frame}

\section{Introduction}

\begin{frame}{Black hole uniqueness}
  \begin{itemize}[<+->]
    \item There exist a plethora of no-hair theorems.
    \item And black hole solutions that evade those theorems in a variety of manners.
    \item Most of those black holes have little to do with astrophysics.
    \item See review by Herdeiro and Radu from 2015.
  \end{itemize}
\end{frame}

\begin{frame}{An example of a no-scalar-hair theorem}
  \begin{itemize}[<+->]
    \item For a Lagrangian of the form:
      \begin{align*}
        \mathcal{L} &= \frac{1}{4}R - \frac{1}{2}\nabla_a \Phi \nabla^a \Phi - V\left( \Phi \right),
      \end{align*}
    \item with the following assumptions:
      \begin{enumerate}
        \item minimally coupled scalar field,
        \item traditional energy conditions on $V$,
        \item scalar field inherits the isometries of the metric.
      \end{enumerate}
  \end{itemize}
\end{frame}

\begin{frame}{Kerr black holes with scalar hair (KBHsSH)}
  \begin{itemize}[<+->]
    \item By violating assumption 3, we find black hole solutions that are:
      \begin{itemize}
        \item asymptotically flat,
        \item regular on and outside an event horizon,
        \item continuously connected to Kerr black holes,
        \item and possess an independent scalar charge.
      \end{itemize}
    \item Violate the assumption such that $T_{ab}$ has the same isometries as the metric.
    \item Kerr black holes with scalar hair \tiny{[Herdeiro, Radu 2014]}
  \end{itemize}
\end{frame}

\section{Kerr black holes with scalar hair}
\subsection{Boson stars}
\begin{frame}
  \tableofcontents[currentsubsection]
\end{frame}


\begin{frame}{Boson stars}
  \begin{itemize}[<+->]
    \item Boson stars are horizonless self-gravitating solitons described by the Einstein-Klein-Gordon equations. \tiny{[Kaup 1968; Ruffini, Bonazzola 1969]}
    \item Can be thought of as the balance between the dispersion and the self-gravity of the scalar field.
    \item We are interested in rotating solutions and they can be described by the following metric and field ansatz:
      \begin{equation*}
        ds^2 = -e^{2F_0}dt^2 + e^{2F_1}\left( dr^2 + r^2d\theta^2 \right) + e^{2F_2}r^2\sin^2\theta\left( d\varphi-Wdt \right)^2, 
      \end{equation*}
      \begin{equation*}
        \Phi = \phi(r,\theta)e^{i\left( m\varphi-wt \right)}
      \end{equation*}
    \item They are only preserved by a single helicoidal Killing vector field:
      \begin{equation*}
        \frac{\partial}{\partial t} + \frac{w}{m}\frac{\partial}{\partial \varphi}
      \end{equation*}
    \item They have a conserved Noether-charge, $Q$, and for boson stars $J=mQ$, where $J$ is the angular momentum of the boson star.
  \end{itemize}
\end{frame}

\begin{frame}{Boson stars}
  \begin{center}
    \includegraphics[width=\textwidth]{boson-stars.pdf}
  \end{center}
\end{frame}

\subsection{Linear analysis}
\begin{frame}
  \tableofcontents[currentsubsection]
\end{frame}

\begin{frame}{Linear analysis: Klein-Gordon equation}
  \begin{itemize}[<+->]
    \item If one considers the Klein-Gordon in the background of Kerr spacetime with the following solution ansatz,
      \begin{equation*}
        \Phi = e^{-iwt}e^{im\varphi}S_{\ell m}(\theta)R_{\ell m}(r).
      \end{equation*}
    \item One finds the Teukolsky equation \tiny{[Teukolsky 1972; Brill et al. 1972]} 
    \item which yields quasi-bound states.
    \item I.e., $w=w_R + i w_I$. Where a critical frequency, $w_c=m\Omega_H$, exists such that
      \begin{itemize}
        \item $w_R>w_c\rightarrow w_I<0$ and the field decays. \tiny{[Degollado et al. 2012]}
        \item $w_R<w_c\rightarrow w_I>0$ and we find superradiant states, i.e. states that grow with time. \tiny{[Press, Teukolsky 1972; Degollado, Herdeiro 2014]}
        \item At the threshold of the superradiant states, $w_R=w_c$, we find true bound states: scalar-clouds. \tiny{[J. Degollado, C. Herdeiro, HR \textit{\ldots to appear \ldots}]}
      \end{itemize}
  \end{itemize}
\end{frame}

\begin{frame}{Scalar clouds around Kerr black holes}
  \begin{center}
    \includegraphics[width=0.8\textwidth]{clouds.pdf}
  \end{center}
\end{frame}

\subsection{Black hole solutions}
\begin{frame}
  \tableofcontents[currentsubsection]
\end{frame}
\begin{frame}{Kerr black holes with scalar hair}
  \begin{itemize}[<+->]
    \item It is ``simple'' to inject a black hole to the center of a boson star:
      \begin{equation*}
        ds^2 = -e^{2F_0}\textcolor{red}{N}dt^2 + e^{2F_1}\left( \frac{dr^2}{\textcolor{red}{N}} + r^2d\theta^2 \right) + e^{2F_2}r^2\sin^2\theta\left( d\varphi-Wdt \right)^2, 
      \end{equation*}
      where $N=1-\frac{r_H}{r}$.
    \item We find the solutions numerically
      \begin{center}
        \includegraphics[width=0.5\textwidth]{HBHs.pdf}
      \end{center}
  \end{itemize}
\end{frame}

\begin{frame}{Kerr black holes with scalar hair}
  \begin{center}
    \includegraphics[width=0.8\textwidth]{HBHs.pdf}
  \end{center}
\end{frame}

\section{$Q$-clouds}

\begin{frame}{$Q$-clouds}
  Discuss $Q$-clouds. Take one of the JCs or seminars I have given and just put it in here?
\end{frame}

\section{Kerr black holes with self-interacting scalar hair}
\begin{frame}
  \tableofcontents[currentsubsection]
\end{frame}
\begin{frame}{Self-interacting scalar hair \\\tiny{C. Herdeiro, E. Radu, HR, PRD 92, 084059 (2015)}}
  Spend 5-10 mins on this part. Add plots etc
  \begin{itemize}[<+->]
    \item Self-interacting boson stars have been shown to alleviate the astrophysically low masses of mini boson stars \tiny{[Colpi et al (1986); Ryan (1996)]}.
    \item By adding positive $\phi^4$ self-interactions to the scalar field, we found generalizations of KBHsSH.
    \item These solutions can be more massive than the non-self-interacting ones.
    \item However, this increase in mass comes solely from the scalar field and not the central black hole.
    \item Therefore, these solutions are ``hairier but not heavier''.
  \end{itemize}
\end{frame}

\section{Kerr black holes with Proca hair}
\begin{frame}{Proca hair \\\tiny{C. Herdeiro, E. Radu, HR, (2016)}}
  Spend 5-10 mins on this part. Add plots, some discussion on the differences and maybe no-hair? probably not\dots
  \begin{itemize}[<+->]
    \item Since Proca fields can ``suffer'' from superradiance around Kerr black holes and rotating Proca stars have been shown to exist \tiny{[Brito, Cardoso, Herdeiro, Radu (2015)]}
    \item we expect Kerr black holes with Proca hair (KBHsPH) to exist.
    \item We did indeed find such solutions.
    \item Their properties are very similar to those of KBHsSH.
    \item One notable difference is that their energy densities possess a second local maximum.
  \end{itemize}
\end{frame}

\begin{frame}{Proca hair \\\tiny{C. Herdeiro, E. Radu, HR, (2016)}}
  \begin{center}
    \includegraphics[width=0.5\textwidth]{proca-w-M.pdf}
    \includegraphics[width=0.5\textwidth]{proca-ro.pdf}
  \end{center}
\end{frame}

\section{Kerr-Newman black holes with scalar hair}
\begin{frame}{Kerr-Newman black holes with scalar hair \\ \tiny{J. Delgado, C. Herdeiro, E. Radu, HR, \ldots\textit{to appear}\ldots (2016)}}
  Spend 5-10 mins on this part. Add plots etc
  \begin{itemize}[<+->]
    \item We have found electrically charged solutions.
    \item Kerr-Newman black holes with gauged/ungauged scalar hair.
    \item We have found that for certain configurations, the scalar hair suppresses properties of the central black hole.
    \item For example, the gyromagnetic ratio of these configurations is $g<2$, only reaching the limit for vanishing hair.
  \end{itemize}
\end{frame}

\section{Conclusions}

\begin{frame}{Conclusions}
  \begin{itemize}[<+->]
    \item Kerr black holes can support scalar hair in classical general relativity.
    \item Their phenomenology can be quite distinct as compared to their Kerr black hole counterparts.
    \item We have already found generalizations of these black holes:
      \begin{itemize}
        \item a scalar field with self-interactions, i.e. $V(\Phi)=\mu^2\Phi^2+\ldots$. We have already considered $\sim\Phi^4$ and $\sim\Phi^6$ terms \tiny{[C. Herdeiro, E. Radu, HR (2015)]}.
        \item scalar field and black hole possess an electric charge \tiny{[J. Delgado, C. Herdeiro, E. Radu, HR (2016)]}
        \item consider other fields, such as Proca fields. \tiny{[C. Herdeiro, E. Radu, HR (2016)]}
        \item These generalizations hint of a more general mechanism where the ``synchronization condition'' $w=m\Omega_H$ generates hairy black holes.
      \end{itemize}
  \end{itemize}
\end{frame}

\end{document}
